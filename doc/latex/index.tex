

This repository is part of the Center of Advanced Robotics by the I\+G\+MR -\/ Aachen University. The main purpose is to define folder structures, coding rules and documentation habbits as a template package. 



\section*{1. Installation}

A summary of install instructions necessary for the template package

\subsection*{1.\+1 Init Submodule}

The \href{https://igm-git.igm.rwth-aachen.de/COAR/igmr_packages/igmr_robotics_system_toolbox}{\tt I\+G\+MR Robotics System Toolbox} is required to compile the ros template nodes. If you haven\textquotesingle{}t already cloned and build the toolbox, you can init and update the submodule\+: 
\begin{DoxyCode}
1 cd perfect\_ros\_repository/ros/modules/igmr\_robotics\_system\_toolbox/
2 git submodule init
3 git submodule update
\end{DoxyCode}
 



$\sim$$\sim$\#\# 1.\+1 Gtest$\sim$$\sim$

$\sim$$\sim$\+N\+O\+T\+ES\+:$\sim$$\sim$ $\sim$$\sim$$\ast$ Simply using {\ttfamily catkin\+\_\+add\+\_\+gtest} in a C\+Make\+Lists.\+txt file (no need for find\+\_\+package(\+G\+Test)) works with G\+Test out of the box. $\sim$$\sim$ $\sim$$\sim$$\ast$ The {\ttfamily libgtest-\/dev} package is already installed by R\+OS.$\sim$$\sim$ $\sim$$\sim$$\ast$ Also note that installing pre-\/compiled G\+Test libraries is discouraged \href{https://github.com/google/googletest/blob/master/googletest/docs/FAQ.md#why-is-it-not-recommended-to-install-a-pre-compiled-copy-of-google-test-for-example-into-usrlocal}{\tt by Google}.$\sim$$\sim$

$\sim$$\sim$\+Gtest is required for building the test cases. Install Gtest as decribed \href{http://ysonggit.github.io/coding/2014/12/19/use-gtest-in-ros-program.html}{\tt here}$\sim$$\sim$

$\sim$$\sim$cd /usr/src/gtest$\sim$$\sim$ $\sim$$\sim$sudo apt-\/get install libgtest-\/dev$\sim$$\sim$ $\sim$$\sim$sudo cmake C\+Make\+Lists.\+txt$\sim$$\sim$ $\sim$$\sim$sudo make$\sim$$\sim$ $\sim$$\sim$\#copy or symlink libgtest.\+a and ligtest\+\_\+main.\+a to /usr/lib folder$\sim$$\sim$ $\sim$$\sim$sudo cp $\ast$.a /usr/lib$\sim$$\sim$ 



\subsection*{1.\+2 Google Benchmark}

Google Benchmark is required for building and running the benchmarks. Install Google Benchmark as described \href{https://github.com/google/benchmark}{\tt here}

In the desired install folder (usually /opt), run the following commands, or paste this into a shell script\+:


\begin{DoxyCode}
1 git clone https://github.com/google/benchmark.git
2 cd benchmark
3 mkdir build
4 cd build
5 cmake .. -DCMAKE\_BUILD\_TYPE=RELEASE
6 make
7 sudo make install
\end{DoxyCode}


\subsection*{1.\+3 Set up G\+CC version alternatives}

To be able to use the latest improvements in the most commonly used C and C++ compiler G\+C\+C/\+G++, such as C++ 14/17 features, install the newest reslease version (e.\+g. currently 7.\+2)\+:


\begin{DoxyCode}
1 sudo add-apt-repository ppa:ubuntu-toolchain-r/test
2 sudo apt-get update
3 sudo apt-get install gcc-7 g++-7
\end{DoxyCode}


{\bfseries Do not delete older versions, unless you know what you are doing!} If you have older versions installed and you want to keep them, set up all possible versions using update-\/alternatives. This will create symbolic links to your installations, so that you can easily switch between them. To see which versions of G\+C\+C/\+G++ are installed on your system, use


\begin{DoxyCode}
1 ls /usr/bin/ | grep ^gcc\(\backslash\)-[0-9]
\end{DoxyCode}


To set up the symbolic links, first remove all previously created links using


\begin{DoxyCode}
1 sudo update-alternatives --remove-all gcc
2 sudo update-alternatives --remove-all g++
\end{DoxyCode}


Then, create new links for all desired versions. You also have to specify a priority for each version. Choose priorities between 0 and 100 depending on your preferences, such that in auto mode, Ubuntu will choose whatever version you desire. As an example, we will assume we installed G\+C\+C/\+G++ 4.\+9 and 7 and want to set 7.\+x as the default version. In this case we can use the following commandos\+:


\begin{DoxyCode}
1 sudo update-alternatives --install /usr/bin/g++ g++ /usr/bin/g++-7 60 --slave /usr/bin/gcc gcc /usr/bin/gcc
      -7
2 sudo update-alternatives --install /usr/bin/g++ g++ /usr/bin/g++-4.9 40 --slave /usr/bin/gcc gcc /usr/bin/
      gcc-4.9
\end{DoxyCode}


By using the \char`\"{}master/slave\char`\"{} setup, we can make sure that whenever a G++ version is chosen, the corresponding G\+CC version is chosen as well. If you want to keep G++ and G\+CC separate, install both update alternatives independently (recommended only for advanced users). The priority is specified following the (master) install, here we chose a priority of 60 for version 7.\+x and 40 for version 4.\+9.

We can now easily switch between versions using


\begin{DoxyCode}
1 sudo update-alternatives --config g++
\end{DoxyCode}


and choosing one of the displayed alternatives. 



\section*{2. Demo}

\subsection*{2.\+1 Run minimal\+\_\+publisher\+\_\+node}

{\bfseries Launch\+:}


\begin{DoxyCode}
1 roslaunch minimal\_publisher\_pkg minimal\_publisher.launch
\end{DoxyCode}
 



\section*{3. Tests}

{\bfseries Build\+:}


\begin{DoxyCode}
1 catkin\_make tests
\end{DoxyCode}


{\bfseries Launch\+:}

To start all available tests, use


\begin{DoxyCode}
1 roslaunch template\_nodes run\_unit\_tests.launch
\end{DoxyCode}


{\bfseries Run\+:}

To start a specific test, use the common syntax used to start a rosnode (using one of the template tests)\+:


\begin{DoxyCode}
1 rosrun template\_nodes conduct\_heavy\_computation\_test
\end{DoxyCode}
 



\section*{4. Benchmarks}

\subsection*{4.\+1 General use}

The basic use-\/pattern for benchmarks is\+:


\begin{DoxyCode}
1 rosrun template\_nodes benchmarks <arguments>
\end{DoxyCode}
 



\subsection*{4.\+2 Running a subset of benchmarks}

Per default, all benchmarks specified in the benchmarks.\+cpp will be run. To run a subset of benchmarks, use\+:


\begin{DoxyCode}
1 rosrun template\_nodes benchmarks --benchmark\_filter= <regex>
\end{DoxyCode}


where \+\_\+\+\_\+$<$regex$>$\+\_\+\+\_\+ defines a regular expression

\subsubsection*{4.\+2.\+1 Regex examples}

Useful examples using regular expressions\+:

\tabulinesep=1mm
\begin{longtabu} spread 0pt [c]{*2{|X[-1]}|}
\hline
\rowcolor{\tableheadbgcolor}{\bf Regular Expression }&{\bf Explanation  }\\\cline{1-2}
\endfirsthead
\hline
\endfoot
\hline
\rowcolor{\tableheadbgcolor}{\bf Regular Expression }&{\bf Explanation  }\\\cline{1-2}
\endhead
\+\_\+\+\_\+\textbackslash{}$^\wedge$$<$name$>$\+\_\+\+\_\+ &Matches benchmarks starting with \+\_\+\+\_\+$<$name$>$\+\_\+\+\_\+ \\\cline{1-2}
\+\_\+\+\_\+$<$name$>$\$\+\_\+\+\_\+ &Matches benchmarks ending with \+\_\+\+\_\+$<$name$>$\+\_\+\+\_\+ \\\cline{1-2}
\+\_\+\+\_\+\textbackslash{}$^\wedge$$<$name$>$\$\+\_\+\+\_\+ &Matches benchmark \+\_\+\+\_\+$<$name$>$\+\_\+\+\_\+ exactly \\\cline{1-2}
\+\_\+\+\_\+$<$name$>$\mbox{[}0-\/9\mbox{]}\{1,{\itshape n}\}\+\_\+\+\_\+ &Matches benchmarks containing \+\_\+\+\_\+$<$name$>$\+\_\+\+\_\+ followed by a number with a length of \+\_\+\+\_\+$\ast$n$\ast$\+\_\+\+\_\+ or less (e.\+g. 0-\/999 for \+\_\+\+\_\+$\ast$n$\ast$\+\_\+\+\_\+ = 3) \\\cline{1-2}
\end{longtabu}




\subsubsection*{4.\+2.\+2 Other useful arguments}

A list of commonly used flags to pass to a benchmarks file\+:

\tabulinesep=1mm
\begin{longtabu} spread 0pt [c]{*2{|X[-1]}|}
\hline
\rowcolor{\tableheadbgcolor}{\bf Benchmark Argument }&{\bf Explanation  }\\\cline{1-2}
\endfirsthead
\hline
\endfoot
\hline
\rowcolor{\tableheadbgcolor}{\bf Benchmark Argument }&{\bf Explanation  }\\\cline{1-2}
\endhead
\+\_\+\+\_\+--benchmark\+\_\+repetitions=$\ast$n$\ast$\+\_\+\+\_\+ &The specified benchmarks are run {\itshape n} times. Reports the individual times, mean, median and standard deviation of the runs \\\cline{1-2}
\+\_\+\+\_\+--benchmark\+\_\+report\+\_\+aggregates\+\_\+only=$\ast$\mbox{[}true/false\mbox{]}$\ast$\+\_\+\+\_\+ &For use with the argument above. When set to true, only mean, median and standard deviation are reported \\\cline{1-2}
\+\_\+\+\_\+--benchmark\+\_\+out\+\_\+format=$\ast$\mbox{[}console/json/csv\mbox{]}$\ast$\+\_\+\+\_\+ &Sets the output format. Per default format is set to console. \\\cline{1-2}
\+\_\+\+\_\+--benchmark\+\_\+out={\itshape filename}\+\_\+\+\_\+ &Save the output in the specified file. Does not supress console output. \\\cline{1-2}
\end{longtabu}




\subsection*{4.\+3 C\+PU scaling warning}

If you see the following warning, when running benchmarks\+:


\begin{DoxyCode}
1 ***WARNING*** CPU scaling is enabled, the benchmark real time measurements may be noisy and will incur
       extra overhead.
\end{DoxyCode}


your machine scales down your C\+P\+U-\/frequency to reduce energy-\/consumption. To disable scaling, use\+:


\begin{DoxyCode}
1 sudo cpupower frequency-set --governor performance
\end{DoxyCode}


The scaling mode is reset on each startup, so you will do no permanent damage

You might need to install linux-\/tools-\/common\+:


\begin{DoxyCode}
1 sudo apt-get install linux-tools-common
\end{DoxyCode}
 

 